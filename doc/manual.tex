\documentclass[a5paper,8pt]{article}
\usepackage[utf8]{inputenc}
\usepackage[IL2]{fontenc}
\usepackage{geometry}
\usepackage[czech]{babel}
\title{IVSKalkulátor\\Uživatelský manuál}
\author{Lidé u výtahu}
\date{\today}
\begin{document}
    \setlength{\parindent}{0em}
    \setlength{\parskip}{1em}
    \maketitle
    
    \newpage

    \section*{Vítejte}
    Děkujeme za pořízení kalkulačního software od skupiny Lidé~u~výtahu. Jsem si jistí, že náš software Vás nezklame. Ba naopak. Náš software prošel sofistikovaným plánování a vývojem, typické pro inženýry vycházející z nejprestižnějších univerzitních institucí, a zaručejeme Vám stoprecentní spokojenost.

    Potřeba počítat je stará, jako písmo\,--\,mnohé nejstarší dochovalé písemnosti mluví o počtech, o daních. Čísla jsou prostě spjatá s lidskou civilizací. Již staří Egypťané vynalezli abakus\,--\,kuličkové počítadloq\,--\,zařízení pro zjednodušení počtů. A jak se velikost lidských společností zvětšovala, tak i její potřeba pro rychlé a efektivní výpočty. Historie moderních početních zařízení počala roku 1642 vynálezem mechanické kalkulačky Wilhelmem Schickerdem. A jak tato historie postupovala, přes logaritmická pravítka, elektromechanické kalkulátory, počítače založené na relé i tranzistorech, tak se dostala do rukou i nám. S radostí Vám představuje nejnovější eveluce ve světe výpočtů.

    A tímto evoluce nekončí, jak vývoj dnešních technoligií postupuje raketově dopředu, tak postupujeme my: náš software je neustále aktualizován a vyvíjen; jsme odhodlaní přidávat funkce a zůstat na vrcholu výpočetního software.
    
    ---Viktor Rucký, zakladatel skupiny Lidé u výtahu
    \newpage
    \section{Instalace}
    TODO
    \section{Spuštění}
    TODO
    \section{Užívání}
    \subsection{Uživatelské rozhraní}

\end{document}